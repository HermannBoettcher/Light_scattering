\documentclass[.../bericht]{subfilies}

\begin{document}

  \chapter{Theoretical background}

    This experiments serves to examine the position fluctuations of polystyrene particles with a diameter of
    \begin{equation*}
      d=\SI{4,28}{\micro\meter}
    \end{equation*}
    in an aqueous dispersion that is highly dilute. To be more precise, the form of the potential surrounding the single particles is to be determined. Moreover, the determination of the dependency of the diffusion coefficient on the distance between particle and the walls of the cuvette containing the dispersion along with the total distance for a given particle are main goals of the experiment. Regarding the used optical tweezers, the light force on the particles is measured. Last, through measurements of silica particles' movement in aqueous dispersions with different concentrations of salt, the variation of the screening length is examined.

    \section{Brownian motion}




\end{document}
